\documentclass[10pt,a4paper]{article}
\usepackage[utf8]{inputenc}
\usepackage[icelandic]{babel}
\usepackage[T1]{fontenc}
\usepackage{amsmath}
\usepackage{amsfonts}
\usepackage{amssymb}
\usepackage{graphicx}
\usepackage{lmodern}
\usepackage{kpfonts}
\usepackage[left=2cm,right=2cm,top=2cm,bottom=2cm]{geometry}
\usepackage{tex4ht}



\begin{document}


\begin{tabular}{lcccccc}
   
      
      \begin{tabular}{lcccccc}
         
         \begin{tabular}{lcccccc}
            
            <FONT size="+2">	extbf{html2latex}</FONT> & 
         \\\end{tabular}
       & \\\end{tabular}
    & \\
\end{tabular}




\begin{tabular}{lcccccc}
   
      
         
         \begin{tabular}{lcccccc}
            \begin{tabular}{lcccccc}
               Navigation & \\
               Home & \\
               Package & \\
               Documentation & \\
               Download & \\
               Screenshots & \\
	       Feedback & \\
               
                  <CENTER>
                  <IMG src="http://sourceforge.net/sflogo.php?group_id=2933&amp;type=1" width="88" height="31" border="0"> </CENTER>
               \\ & 
            \end{tabular}
         \\ & \end{tabular}
       & 
     
         \begin{tabular}{lcccccc}
            \begin{tabular}{lcccccc}
\huge{html2latex}

\LARGE{Summary}

html2latex is a Perl script designed to convert a properly formatted
HTML file into a properly formatted LaTeX file.

\LARGE{News}

Version 1.0 is out. It basically is a installation fix for 0.9, but it
also adds the 'kill' tag type which allows you do such things as
remove any javascript.  'make test' failed in 0.9, which could be a
major headache for some people.  Version 0.9 is a minor release that
supports international characters, quote-expansion, plus a fex bug
fixes.  You can dowload the latest tar.gz <A
href=ftp://html2latex.sourceforge.net/pub/html2latex/html2latex-latest.tar.gz>here.
If you already got 0.9 installed and aren't bothered by javascript, you don't have to bother with 1.0; it's just the same.

\LARGE{Supports}
\begin{enumerate}
   \item It can handle URLs on the command line and in the IMG tag.
   \item Converts pictures from jpeg or gif to png.  pdflatex can have included pngs.
   \item Renders nested tables correctly.
   \item Supports most international characters (umlats, accents, etc).
   \item Converts all headers into sections.  This can be easily customized.
   \item Lists of any form.
   \item Endless configuration thourgh command-line options or an XML config file.
   \item It is also very easy to extend by writing your own handlers.
\end{enumerate}


\LARGE{Feedback}
If you try out the software, please go to the feedback site and take the
survey.  Or you can put comments in the
forum, or <A
href="mailto:peterthatcher@asu.edu">email me. I'd like your suggestions.

\LARGE{Site Directions}
\begin{enumerate}
\item Home - Here
\item Package - Link to unzipped files of latest files.  Look here for
ChangeLog, TODO, README, etc.
\item Documentation - Right now, a man page.
\item Download - Sight listing all releases.
\item Screenshots - Take a look at what html2latex can do.
\item Feedback - Please, fill out a survey and tell me what you think.
\end{enumerate}

\LARGE{Requirements}
All required modules listed below and all of their dependencies can be found here
<P>
html2latex requires the following modules for basic operation:
\begin{enumerate}
\item HTML::Tree - It requires HTML::Parser.
\item XML::Simple - It requires XML::Parser.
\end{enumerate}
html2latex can use the following moduls for advanced operation:
\begin{enumerate}
\item LWP::Simple - Used do download URLs.  Requires lots of things; look for Bundle::LWP or libwww.
\item URI - Comes with libwww or Bundle::LWP.  Also required to grab URLs.
\item Image::Magick - If you want to convert images to PNGs.
\end{enumerate}

<P> The easiest way to get these modules is to use the CPAN module.
Try 	extbf{man CPAN}.



                & \\
            \end{tabular}
         \\ & \end{tabular}
       & 
   \\
\end{tabular}


\end{document}