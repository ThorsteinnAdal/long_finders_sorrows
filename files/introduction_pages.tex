\replaceCompany[%(CompanyName)s
] will operate its fleet as efficiently in respect to energy consumption while complying with all applicable regulations. In order to achieve the corporate energy reduction goals and simultaneously reduce the operating costs, \replaceCompany[%(CompanyName)s
] uses appropriate programs to reduce energy and other utility costs; procure energy-efficient products; construct, operate, and
maintain energy-efficient ships; and promote efficient use of energy among Eimskip employees. In
support of the Eimskip program the company maintains a Environmental and Efficiency
Management Plan which is on corporate level and Ship Energy Efficiency Management Plan
(SEEMP) on ship specific level, that addresses the core program elements listed in the following
section.

The mission of the company's project is to use the SEEMP plan to comply with Regulation 22 of
Annex VI of the MARPOL Convention, and to use systematic methods and tools to improve the
energy efficiency and reduce fuel consumption of our fleet.
To assist in achieving the goals, the Marorka energy management program is intended to improve
the energy efficiency of the fleet.

Ship Energy Efficiency Management Plan (SEEMP) is mandatory for ships over 400 GT from the
1st of January 2013 in accordance. All ships subject to the regulation must keep onboard a
SEEMP outlining efforts to increase efficiency and reduce regulations. The SEEMP is an official
ship document and may be required to be presented to Port State Control in case of inspections
and surveys.

A SEEMP, which reflects the 2012 Guidelines for the Development of Ship Energy Efficiency
Management Plan MEPC.213(63), has been prepared according to Regulation 22 of Annex VI of
the International Convention for the Prevention of Pollution from Ships, 1973, as modified by the
Protocol of 1978 relating thereto (MARPOL 73/78) (hereafter referred to as “MARPOL”).
The SEEMP will be used for monitoring ship efficiency performance over time and some options
to be considered when seeking to optimize the performance of the ship. This SEEMP is a part of
a company-wide energy and environmental management plan encompassing the fleet in its
entirety.
The purpose of a SEEMP is to establish a mechanism for the vessel to improve the energy
efficiency of a ship’s operation. The SEEMP seeks to improve a ship’s energy efficiency through
four steps: planning, implementation, monitoring, and self-evaluation and improvement.